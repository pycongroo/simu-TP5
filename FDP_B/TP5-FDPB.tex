%% LyX 2.0.8.1 created this file.  For more info, see http://www.lyx.org/.
%% Do not edit unless you really know what you are doing.
\documentclass[english]{article}
\usepackage[latin9]{inputenc}
\usepackage{esint}
\usepackage{babel}
\begin{document}
Por dato $f(x)$ es lineal

$f(x)=mx+b$

Obtenemos $b$:

$f(45)=3f(15)$

$45m+b=3(15m+b)$

$45m+b=45m+3b$

$b=0$

Por propiedad de sumatoria de probabilidades:

$\intop_{15}^{45}f(x)dx=1$

Despejamos $m$:

$\intop_{15}^{45}mxdx=1$

$(\frac{mx^{2}}{2})_{15}^{45}=1$

$\frac{m}{2}(45^{2}-15^{2})=1$

$\frac{m}{2}1800=1$

$m=\frac{1}{900}$

Reemplazamos y obtenemos la funcion $f(x)$:

$f(x)=\frac{x}{900}$

Calculamos la funci�n de probabilidad acumulada:

$F(z)=\intop_{15}^{z}f(x)dx$

$F(z)=\intop_{15}^{z}\frac{x}{900}dx$

$F(z)=(\frac{x^{2}}{1800})_{15}^{z}$

$F(z)=\frac{(z^{2}-15^{2})}{1800}$

Calculamos la funcion inversa de probabilidad acumulada:

$F(x)=\frac{(x^{2}-15^{2})}{1800}=y$

Despejamos $x$, ya que $x=F^{-1}(y)$

$x^{2}-15^{2}=1800y$

$x^{2}=15^{2}(8y+1)$

$F^{-1}(y)=x=15\sqrt{8y+1}$
\end{document}
